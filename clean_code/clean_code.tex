\documentclass[10pt]{article}
\usepackage{amsmath}
\usepackage{amssymb}
\usepackage{algorithm}
\usepackage{algorithmic}
\usepackage{mathtools}
\usepackage{graphicx}
\usepackage{listings}

\title {On code quality from a maintenance viewpoint}
\begin{document}
\maketitle


\begin{section}{Configuring vim}
The attached configuration in .vimrc is minimalistic and pragmatic:

\begin{lstlisting}
:imap jj <Esc>
set expandtab ts=4 sw=4 ai
\end{lstlisting}


The command imap replaces the use of the esc key with a double pressing of j.
Expandtab converts tab spaces in N blank spaces, while ts=4 set this N equal
to 4. The auto-indention is given by ai with a width of value sw=4.
Typing :retab ensures to replace
the tabs with whitespaces in the current (previously created) file.
\end{section}


\begin{section}{Pylint and Pytest}
Writing clean code is essential: not only from an algorithmic viewpoint 
(performance), but also w.r.t. readability (maintenance).


The use of pylint gives a good insight on your code readability quality,
and its straightforward use via \emph{pylint pythonfile.py}
prints on screen always interesting suggestions.


Speaking again about maintenance, every time you implement a new function, 
in principle you should also
write a test to check that this function works as expected 
(up to some reasonable level).
For example you can test against some trivial conditions, easy
cases or extreme values on which the code has a known output.
Writing tests can seem to be time consuming, but it is
very convenient in the long run: every time you modify some parts of your code,
if the tests are still successful, you have some chance to believe you didn't
damage your work.


In python you can use pytest. Running it on a script will just run
all functions whose name start with the word 'test' followed by an underscore.
Remember that each function of such a kind should end with an assert condition.
See the attached code for a minimalistic example.
\end{section}


\end{document}
