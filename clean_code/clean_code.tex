\documentclass[10pt]{article}
\usepackage{amsmath}
\usepackage{amssymb}
\usepackage{algorithm}
\usepackage{algorithmic}
\usepackage{mathtools}
\usepackage{graphicx}

\title {template}
\author {me}

\begin{document}

\maketitle
\begin{section}{Configuring vim}
The following configuration in .vimrc is minimalistic and pragmatic:


:imap jj <Esc>
set expandtab ts=4 sw=4 ai


The command imap replaces the use of the esc key with a double pressing of j.
Expandtab convers tab spaces in N blank spaces, while ts=4 set this N equal
to 4. The autoindention is given by ai with a witdh of value sw=4.
Typing :retab ensures to replace
the tabs with whitespaces in the current (previously created) file.
\end{section}


\begin{section}{Pylint and Pytest}
Writing clean code is essential: not only from an algorithmic viewpoint 
(performance), but also regarding readability (maintenance).


The use of pylint gives a good insight concerning the quality of the code
w.r.t. this last criteria, and its use is traighforward: just type
'pylint pythonfile.py' and you'll get an output of suggestions.


Every time you implement a new function, in principle you should also
write a test to check that this function works (up to some reasonable level)
as expected. For example you can test against some trivial conditions, easy
cases or extremal values on which the code has an easy-to-compute
behavior. Writint tests can seem at a first sight time consuming, but it is
very convenient in the long run. Everytime you modify some parts of your code,
if the tests are still successful, you have some chance to believe you didn't
damage your work.


In python you can use pytest. Running pyunit on a script will just run
all functions whose name start with the word 'test' followed by an underscore.
Remember that each function of such a kind should end with an assert condition.


\end{section}


\end{document}
