\documentclass[10pt]{article}
\usepackage{amsmath}
\usepackage{amssymb}
\usepackage{algorithm}
\usepackage{algorithmic}
\usepackage{mathtools}
\usepackage{graphicx}
\usepackage{listings}

% if you want to reduce the space from the top
%\usepackage[scale=0.75,top=1cm]{geometry} 

\title {My main resources for learning Python}
%\author {me}

\begin{document}

\maketitle

\begin{itemize}

\item{\textbf{Learning Python}}:
an extensive, meticolous exposition of essentially everything concerning
the basics. Very long, hard to digest, but effective;

\item{\textbf{Python Workout}}:
the opposite approach. Collection of 50 exercises straigh-to-the-point
to improve the understanding by direct experience.
Covering from mutables, immutables, functions, objects, itereators...

\item{\textbf{Python Flashcards}}:
a quick check of the basic background;

\item{\textbf{Efficient Python}}:
covering stuff like pylint, parallelization and interfacing C.
Develop more sensitivity for memory usage and processor load.
Turned on an interest for Computer Architectures and Graphic Cards;

\item{\textbf{Machine Learning Lab}}:
internal Academic reading to learn scientific-oriented stuff like numpy, scipy,
and the Fundamentals of Machine Learning;

\item{\textbf{Nvidia Certificate -
 Fundamentals of Accelerated Computing with CUDA Python}}:
a short online course (few hours) on how to use
Numba for CUDA parallelization with Numpy;

\item{\textbf{Coursera Certificate - IBM AI Engineering}}:
improving the familiarity with classic Machine Learning 
libraries and applications, notably PyTorch;

\item{\textbf{Python Certificate - PCEP-30-01}}:
a way to officially confirm by understanding of basic Python.
The more advanced certificates tend to focus on user interfaces
and networking, so I decided to stop here.

\end{itemize}
\end{document}
