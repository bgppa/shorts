\documentclass[10pt]{article}
\usepackage{amsmath}
\usepackage{amssymb}
\usepackage{algorithm}
\usepackage{algorithmic}
\usepackage{mathtools}
\usepackage{graphicx}
\usepackage{listings}
\usepackage[scale=0.75,top=1cm]{geometry}

\title {A simple Monte Carlo parallelization in numpy}
\begin{document}

\maketitle

\begin{section}{Multiple processors}
After importing \emph{psutil}, in Python3 use:
\begin{lstlisting}
psutil.cpu_count(logical = True)  # counts total virtual CPUs
psutil.cpu_count(logical = False) # only physical CPUs
\end{lstlisting}

In general it is believed that using all the virtual cpus gives a speedup
of +30\% w.r.t. using only all the physical one. Take it with a grain of salt
and always remember to keep an eye on RAM memory usage.
\end{section}


\begin{section}{Parallelizing in python with multiprocessing}
In few words, it's the \emph{map} builtint spreaded across
multiple subprocesses:
\begin{lstlisting}
# Displays [1, 2, 3]
from multiprocessing import Pool        # Standard parallel library

def f(x_int):
    return x_int + 1

pool = Pool(processes = 3)              # run 3 processes in paral.
argument_list = [0, 1, 2]               # prepare to run f(0), f(1), and f(2)
results = pool.map(f, argument_list)    # run f(0),.., store the results here
print(results)
\end{lstlisting}

\end{section}


\begin{section}{Parallelization in Monte Carlo}
MC estimations consist of averaging among
multiple simulated random variables assumed to be 
independent and equally distributed (so to apply the central limit theorem
and compute confidence intervals).

If parallelization in itself is a complex topic, in the case
of MC it simplify due to the fact of having
completely independent processes (no need to send messages between them,
i.e. no MPI, etc...).

Just remember to initialize the seed on every subprocess differently,
to avoid having the \emph{same} sequence of pseudorandom numbers 
on every subprocess.
The attached code contains a complete implementation of pi estimation.


For an alternative approach, you can try with Numba to achieve CPU
and GPU parallelization - we'll deal with it on another note.
\end{section}

\end{document}
